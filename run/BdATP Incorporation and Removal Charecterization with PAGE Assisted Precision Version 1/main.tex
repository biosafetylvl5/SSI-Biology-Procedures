\documentclass{ssiBio}
\usepackage{multirow}
\usepackage{amssymb}
\title{\BdATP{} Incorporation and Removal Characterization with PAGE Assisted Precision Version 1} % CHANGE THIS
\author{Written by \textbf{Michael Uttmark}\\ % CHANGE THIS 
		Not Peer Reviewed %Checked by \textbf{}\\ % CHANGE THIS
        For the Stanford Student Space Initiative Biology Sub team}
\date{\textbf{Written:} November 8, 2017 \,\textbf{Performed:} November 8, 2017 \,\textbf{Printed:} \currenttime{}, \today{}}

\begin{document}

\maketitle
\section{Procedure Purpose} % CHANGE THIS
Determine if the modified nucleotide, \BdATP{}, can be noticeably incorporated and then removed by \tdt{} in "standard conditions".
\begin{figure}[ht]
\centering
\includegraphics[width=2in]{./resources/BdATP-Structure.png}
\caption{\BdATP{}}
\label{bdatp}
\end{figure}
\section{Overview} % CHANGE THIS 
This lab will attempt to append \BdATP{} to a \textbf{short (25bp) primer}. The effectiveness of this attempt will be determined by attempting to form a homopolymer on the modified primer. If a homopolymer is formed, the blocking groups did not effectively prevent their formation. Moreover, another sample will have their blocking groups removed by ultraviolet light (365nm) and the will be treated with the same dNTP extension process as the "blocked" sample. This could be due to many reasons (the most likely of which being that the blocking groups either (1) were appended without the 2' nitrobenzyl due to sample degradation or (2) were not appended). If the homopolymer was not formed (but a homopolymer was formed on the controls) it follows that the blocking groups prevented the formation of the homopolymer, likely due to them preforming their intended function. Moreover, all samples will be run on a PAGE gel to achieve single nucleotide resolution. This will allow us to confirm that the \BdATP{} is the only base appended to the "blocked" sample. A ddATP control will help determine the effectiveness of the blocking group as a positive control for effective blocking.

% Safety First! ALSO, % CHANGE THIS
\section{Safety Information}
\begin{safety}
\begin{enumerate}
\SYBRGOLD{} % For select reagents, I've created custom commands, so we don't have to copy and paste every time :)
\tdtSafety{} 
\tdtBufferSafety{}
\item{Working in a communal lab space is dangerous. Do not assume your fellow workers cleaned up sufficiently}
\end{enumerate}
\end{safety}

\section{Materials}
\begin{itemize}
\item{Primer (25bp)}
\item{100mM \BdATP{} Stock}
\item{10mM dNTP Stock}
\item{100mM dATP Stock}
\item{10mM ddATP Stock}
\item{5X \tdt{} Buffer}
\item{\tdt{} Stock (20U/\uL{})} % Should REMEMBER TO CHANGE THIS to 15U/uL - we have much more of this available 
\item{Nuclease Free Water}
\item{TBE Buffer}
\item{20\% Urea Denaturing Gels}
\item{SYBR Gold}
\item{365nm UV Radiation Source (We will be using our "UV Death Chamber")}
\end{itemize}
% Now for the _good_ stuff  

\section{Procedure}% CHANGE THIS
\subsection{Sample Preparation}
\begin{figure}[ht]
\begin{center}
\begin{tabular}{ c | l }
	\hline
	A & The primer incubated with \textbf{dATP} and then commercial \textbf{dNTPs} \\
	B1 & The primer incubated with \textit{just} \textbf{NBdATP} \\
	B2 & The primer incubated with \textbf{NBdATP} and then \textbf{dNTPs}\\
	B3 & The primer incubated with \textbf{NBdATP}, irradiated with 365nm ultraviolet light, and then \textbf{dNTPs}\\
	B1*, B2* B3* & A repeat of B1, B2 and B3 prepared separately from B1, B2 and B3\\
	C & The primer incubated with just \textbf{dNTPs} in the second incubation \\
	D & The primer incubated with \textbf{ddATP} nucleotides and then \textbf{dNTPs} \\
	X & The primer incubated with \textbf{dNTPs} but \textbf{no \tdt{}} in the second incubation \\
	\hline
\end{tabular}
\end{center}
	\label{sampleTable}
	\caption{Samples and their experimental conditions}
\end{figure}
\begin{enumerate}
\item{Remove \BdATP{}, \tdt{}, primer, \tdt{}  buffer, ddATP stock and dATP stock from -20\C{} freezer}
\item{Let \BdATP{} thaw on ice in dark}
\item{Other reagents can thaw on ice in the light}
\subsection{Attempted blocking}
\item{Label four PCR Tubes A, B, B* and D, respectively}
\item{Pipette 10.6\uL{} of nuclease free water into tube A}
\item{Pipette 19.8\uL{} of nuclease free water into tube B and B*}
\item{Pipette 11\uL{} of nuclease free water into tube D}
\item{Pipette 4.0\uL{} 5X \tdt{} reaction buffer into PCR Tubes A and D}
\item{Pipette 6.0\uL{} 5X \tdt{} reaction buffer into PCR Tube B and B*}
\item{Dilute Nucleotides:
\begin{enumerate}
\item{Label a PCR Tube "dATP Dilute"}
\item{Pipette 9\uL{} of nuclease free water into PCR Tube}
\item{Pipette 1\uL{} of dATP stock into PCR Tube}
\item{Vortex directly before use}
\end{enumerate}
}
\item{Pipette 0.5\uL{} of primer into PCR Tubes A and D}
\item{Pipette 0.75\uL{} of primer into B and B*}
\item{Pipette 3\uL{} of dATP dilute into PCR Tube A}
\item{Pipette 0.45\uL{} of \BdATP{} stock into PCR Tube B and B*}\\ % Should place this _after_ the ddATP step - we want the BdATP to be out for as short as possible
\item{Pipette 3\uL{} of ddATP \textbf{10mM stock} into PCR Tube D}
\item{Gently pipette 2\uL{} \tdt (15/\uL{}) into PCR tubes A and D.}
\item{Gently pipette 4\uL{} \tdt (15/\uL{}) into PCR tube B and B*.}
\item{Incubate samples at 37\C{} for 30 minutes}
\item{Return dATP and ddATP to -20\C{} freezer.}
\stopPoint{} 
\subsection{Extending}
Based off our standard \tdt{} extending procedure \cite{genTdT}.
\item{Label two PCR Tubes C and X, respectively}
\item{Pipette 10.5\uL{} nuclease free water into PCR Tube C (see above, \textbf{ATTEMPTED BLOCKING})}
\item{Pipette 12.5\uL{} of nuclease free water into PCR Tube X (see above, \textbf{CONTROLS})}
\item{Pipette 0.5\uL{} of primer into both PCR Tubes}
\item{Pipette 4.0\uL{} 5X \tdt{} reaction buffer into PCR Tube C and X}
\item{Pipette 3\uL{} of dNTP stock into PCR Tubes C and X}
\item{Set PCR Tube X aside.}
\item{Wait until the previous samples have finished incubating}
\item{Label two PCR Tubes "B1" and "B1*}
\item{Label two PCR Tubes "B3" and "B3*}
\item{Relabel PCR Tubes B and B* as B2 and B2* respectively}
\item{Pipette 10\uL{} from B2 into B1}
\item{Pipette 10\uL{} from B2* into B1*}
\item{Pipette 10\uL{} from B2 into B3}
\item{Pipette 10\uL{} from B2* into B3*}
\item{Pipette 2\uL{} EDTA into \textbf{B1 and B1*} to stop the reaction \cite{Invitrogen2002}}\\
\item{Pipette 2\uL{} nuclease free water into B1 and B1*}
\item{Place B1 and B1* into -20\C{} freezer for later use}
\item{Expose B3 and B3* to 365nm of ultraviolet light for 15 minutes with UV deathchamber, or 30 minutes with flashlight if chamber is not available}\\
	B3* was irradiated first at 80mA for 15 minutes and B3 was irradiated at 200mA for 15 minutes.
\item{Pipette .4\uL{} of dNTP stock into PCR Tubes \textbf{A}}
\item{Pipette .4\uL{} of dNTP stock into PCR Tubes \textbf{B2}}
\item{Pipette .4\uL{} of dNTP stock into PCR Tubes \textbf{B3}}
\item{Pipette .4\uL{} of dNTP stock into PCR Tubes \textbf{B2*}}
\item{Pipette .4\uL{} of dNTP stock into PCR Tubes \textbf{B3*}}\\
	Please check these off as you go - it's easy to miss one.
\item{Gently pipette 2\uL{} \tdt{} (20 U/\uL{}) into \textbf{PCR Tube C} and \textbf{B3} along with \textbf{B3*}}
\item{Incubate \textbf{all} samples \textbf{except B1 and B1*} at 37\C{} for 30 minutes}
\item{Wait until the samples have finished incubating.}
\item{Stop any \tdt action by adding 2\uL{} 0.5M EDTA to the all PCR tubes \textbf{except all B and B*} after incubation.\cite{Invitrogen2002}}
\item{Pipette 1\uL{} 0.5M EDTA into B2 and B2*}
\item{Pipette 1\uL{} 0.5M EDTA into B3 and B3*}

\RstopPoint{} 
\subsection{Analysis}
\subsubsection{XCell Surelock Setup and Pre-Run}
\item{Remove 20\% polyacrylamide gel from pouch and rinse with deionized water.}
\item{Peel off tape on bottom of 20\% polyacrylamide gel and remove the comb.}
\item{Lower the Buffer Core (the piece that holds the gels) into the Lower Buffer Chamber so that the negative electrode fits into the opening in the gold plate.}
\item{Insert the Gel Tension Wedge into the XCell Surelock behind the buffer core. Make sure it is in its 'unlocked' position, which allows the wedge to slip into the unit.}
\item{Insert gel cassettes into the lower buffer chamber. The shorter "well" side of the cassette faces into the buffer core. The slot on the back must face outward. If only one gel is being run, insert a buffer dam in the place of a gel cassette.}
\item{Pull forward on the Gel Tension Lever toward the buffer core until the gel cassettes are snug against the buffer core. This puts it in the 'locked' position.}
\item{Fill the Upper Buffer Chamber (between the gels) with running buffer. Ensure it is not leaking.}
\item{Fill the Lower Buffer Chamber completely with running buffer by pouring TBE next to the Gel Tension Wedge.}
\item{Pipette 12\uL{} of running buffer into each gel well.}
\item{Place the gel cover on the apparatus in the correct orientation. Connect the electrodes to the power source, and pre-run the gel for 30 minutes at 150V.}
\item{When there is only 5 minutes left on the incubation, retrieve sample B1 and B1* from the freezer and let thaw on ice}
\subsubsection{Run Gel}
\begin{figure}[ht]
\begin{center} 
\begin{tabular}{|l|l|}
\hline
Well number				& Sample \\ \hline
\rownumber                                 & 10/60 DNA Ladder   \\ \hline
\rownumber				   & Custom Ladder (40ng) \\ \hline
\rownumber                                 & B1 (40ng)\\ \hline
\rownumber                                 & B2 (40ng)\\ \hline
\rownumber                                 & B3 (40ng)\\ \hline
\rownumber                                 & B1* (40ng)\\ \hline
\rownumber                                 & B2* (40ng)\\ \hline
\rownumber                                 & B3* (40ng)\\ \hline
\rownumber                                 & X (40ng) \\ \hline
\rownumber				   & D (40ng)	\\ \hline
\rownumber                                 & C \textbf{From last procedure} (40ng)	\\ \hline
\rownumber                                 & A (40ng)	\\ \hline
\rownumber                                 & C (40ng)  \\ \hline
\rownumber                                 & Custom Ladder (40ng)     \\ \hline
\rownumber                                 & 10/60 DNA Ladder  \\ \hline
% Note to procedure writer - it's a good practice to include a ladder in the middle of the gel to prevent symmetry and to enable smiling analysis. If the ladders are only on the end, we don't get a good picture of the extent of any smiling that took place.
\end{tabular}
\label{tab:samples} %Label your stuff, this is for referencing in the future
\caption{Wells and their assorted reagents} %Caption!
\end{center}
\end{figure}
\textbf{Note}: Be relatively swift about mixing and loading, as the samples will gradually begin to evaporate if left on the parafilm for too long.
\item{Obtain a sizable piece of parafilm. Pipette 3 \uL{} of 2X Gel Loading Dye in a row of 15 droplets.}
\item{For the 10/60 Ladder samples, pipette 1uL of 10/60 Ladder and 4 uL of running buffer and mix.}
\item{For the remaining droplets, add 3 \uL{} of the appropriate sample. See the corresponding table (Figure \ref{tab:samples}) above for sample location and order.}
\item{As you go, pipette up and down to mix thoroughly.}
\item{Load the gels (with 5 \uL{} sample in each well) when they are finished pre-running. Ensure pipette tip is fully in the well, and depress slowly and carefully. Work quickly to minimize diffusion.}
\item{Run the gel(s) at 150V until the dark blue dye is at the bottom.}

\subsubsection{Stain \& View Gel}
\item{While the gel runs, prepare 1X SYBR Gold Staining Solution with TBE as dilute}
	\begin{enumerate}
		\item{Add 6\uL{} SYBR Gold to 60\uL{} of TBE running buffer}
	\end{enumerate}
\item{Once gel has finished running, \textbf{\textit{lightly}} agitate gel while submerged in solution for 60 minutes.}
\item{Review gel with gel viewer. Until unnecessary, place gel back in stain for 20-minute increments and re-image.}
\item{Post pictures to Slack.}\\
\end{enumerate} 

% Stop Procedure
\section*{Stop Procedure}
\begin{enumerate}
\item{Pipette samples into PCR tubes if not already contained in an appropriate manner}
\item{Label containers if not already labeled}
\item{Freeze samples at -20\C{}}
\end{enumerate}
\newpage
\section{Analysis}
\begin{figure}[ht]
	\begin{center}
		\includegraphics[width=.5\textwidth]{./resources/gels/labeled.png}
		\caption{PAGE Gel used to determine length of samples}
		\label{mainGel}
	\end{center}
\end{figure}
In this gel we have the samples enumerated in Figure \ref{mainGel} above. L is used as an abbreviation for the 10/60 DNA Ladder and CL is used as an abbreviation for the Custom Ladder prepared previously. Below, we have a comparison of the samples with our predicted results:
\newcommand{\Y}{\textbf{\checkmark}}
\newcommand{\B}{\textbf{B}}
\newcommand{\N}{\textbf{$\times$}}
\begin{figure}[ht]
\centering
\label{tab:sampleExtension}
\begin{tabular}{llll}
	Sample              & Predicted if Blocking Occurred & Predicted if No Blocking Occurred & Observed $\frac{\textrm{B}}{\textrm{B*}}$\\\hline
\multirow{2}{*}{B1} & \multirow{2}{*}{\B}             & \multirow{2}{*}{\N}                & \N         \\
                    &                                 &                                    & \N        \\\hline
\multirow{2}{*}{B2} & \multirow{2}{*}{\B}             & \multirow{2}{*}{\Y}                & \Y         \\
                    &                                 &                                    & \Y         \\\hline
\multirow{2}{*}{B3} & \multirow{2}{*}{\Y}             & \multirow{2}{*}{\Y}                & \Y         \\
                    &                                 &                                    & \Y         \\\hline
X                   & \N                              & \N                                 & \N         \\\hline
D                   & \B                              & \B                                 & \B         \\\hline
A                   & \Y                              & \Y                                 & \Y         \\\hline
C                   & \Y                              & \Y                                 & \Y        
\end{tabular}
	\caption{Samples and Extension}
\end{figure}
All samples from this experiment support the conclusion that \textbf{blocking did not occur}. However, the past procedures both contradict and support this conclusion. Below, we have a similar table that includes the past procedures with relevant data:

\begin{figure}[ht]
\centering
\label{tab:sampleExtensionAll}
\begin{tabular}{lllllll}
	\multirow{2}{*}{Sample} & \multirow{2}{*}{Predicted if Blocking Occurred} & \multirow{2}{*}{Predicted if No Blocking Occurred} & \multicolumn{4}{c}{Observed $\frac{\textrm{B}}{\textrm{B*}}$}\\
                        &                                                 &                                                    & Agarose  & 4  & 5  & Current \\\hline
B1                      & \B                                              & \N                                                 &          &    & \N & \N      \\
                        &                                                 &                                                    &          &    & \N & \N      \\\hline
\multirow{2}{*}{B2}     & \multirow{2}{*}{\B}                             & \multirow{2}{*}{\Y}                                & \B / \N  & \Y & \Y & \Y      \\
                        &                                                 &                                                    &          &    & \N & \Y      \\\hline
\multirow{2}{*}{B3}     & \multirow{2}{*}{\Y}                             & \multirow{2}{*}{\Y}                                &          &    &    & \Y      \\
                        &                                                 &                                                    &          &    &    & \Y      \\\hline
X                       & \N                                              & \N                                                 & \N       & \N & \N & \N      \\\hline
D                       & \B                                              & \B                                                 &          & \B & \B & \B      \\\hline
A                       & \Y                                              & \Y                                                 & \Y       & \Y & \Y & \Y      \\\hline
C                       & \Y                                              & \Y                                                 & \Y       & \Y & \Y & \Y    
\end{tabular}
	\caption{Samples and Extension Across Multiple Experiment}
\end{figure}
As one can see from the results in the table above in Figure \ref{tab:sampleExtensionAll}, our results have not been entirely consistent. The results seem to currently favor the conclusion that we are not blocking successfully. This is concerning, not simply because blocking is our goal, but because this experiment has already been preformed by another group with great success. Moreover, the only consistent results come from B1, who (with the exception of B3) has the least repetitions. 

Our solution to this issue of repeatability is the (1) review our past experiments and look for changes and (2) run a "mega-replicate" experiment that puts our focus on solely B1 and B2, allowing us to do many replicates and nail down what is really happening.

\begin{figure}[ht]
	\begin{center}
		\includegraphics[width=.95\textwidth]{./resources/gels/labeled.png}
		\caption{PAGE Gel used to determine length of samples (Large)}
		\label{mainGel-big}
	\end{center}
\end{figure}
\bibliographystyle{ieeetr}
\bibliography{main}
\end{document}
